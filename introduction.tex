\chapter{Introduction}

This is an example using the \LaTeX{} template for UCI theses and
dissertation documents \cite{uci-thesis-latex}. Figure
\ref{fig:sourcecode} is just for illustration purposes, as is Table
\ref{tab:coordinates}.

\begin{figure}
\begin{verbatim}
#include <iostream>
int main(int argc, char** argv) {
  std::cout << "Hello World." << std::endl;
  return 0;
}
\end{verbatim}
  \caption{Example source code.}
  \label{fig:sourcecode}
\end{figure}

\section{Background}

Sources: Dark Energy Task Force Report (Albrecht et. al)

Lorem ipsum dolor sit amet, consectetur adipisicing elit, sed do
eiusmod tempor incididunt ut labore et dolore magna aliqua. Ut enim ad
minim veniam, quis nostrud exercitation ullamco laboris nisi ut
aliquip ex ea commodo consequat. Duis aute irure dolor in
reprehenderit in voluptate velit esse cillum dolore eu fugiat nulla
pariatur. Excepteur sint occaecat cupidatat non proident, sunt in
culpa qui officia deserunt mollit anim id est laborum.

\begin{table}
  \centering
  \begin{tabular}{|rr|r|}
    \hline
    $x$ & $y$ & $z$ \\
    \hline
    14 & 12 & -2 \\
    0 & 33 & -25 \\
    -3 & 11 & 22 \\
    4 & 4 & 6 \\
    \hline
  \end{tabular}
  \caption{Example coordinates.}
  \label{tab:coordinates}
\end{table}

Lorem ipsum dolor sit amet, consectetur adipisicing elit, sed do
eiusmod tempor incididunt ut labore et dolore magna aliqua. Ut enim ad
minim veniam, quis nostrud exercitation ullamco laboris nisi ut
aliquip ex ea commodo consequat. Duis aute irure dolor in
reprehenderit in voluptate velit esse cillum dolore eu fugiat nulla
pariatur. Excepteur sint occaecat cupidatat non proident, sunt in
culpa qui officia deserunt mollit anim id est laborum.

\section{History}
\section{Dark energy}
\section{Theoretical framework}
\section{Stage IV Dark Energy Experiments}
\subsection{Vera C. Rubin Observatory}
\subsection{Dark Energy Spectroscopic Instrument}
\section{Observational probes}
\subsection{Weak gravitational lensing}
\subsection{BAO in the Lyman-$\alpha$ forest of high redshift quasars}
\section{Motivation and results}
% talk about requirements here (both in terms of image quality and electrical)
% linearity requirements are motivated by requirements on photometric accuracy

%%% Local Variables: ***
%%% mode: latex ***
%%% TeX-master: "thesis.tex" ***
%%% End: ***
